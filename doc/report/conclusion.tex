\subsection{Bilan personnel}
Pour conclure, ce projet nous a permis d'apprendre et mettre en pratique de nombreux concepts propres à la programmation Android. Nous nous sommes également familiarisé avec le langage Kotlin, qui, une fois pris en main, simplifie et optimise grandement les opérations qui peuvent être plus complexes et fastidieuses en Java. Enfin, nous restons sur un sentiment très satisfaisant de cette première expérience dans le monde du développement Android.

\subsection{Problèmes rencontrés}
Les principales difficultés rencontrées durant le développement de ce projet sont les suivantes :
\begin{itemize}
    \item Prise en main de la documentation Android officielle : par moments des raccourcis sont pris dans les explications, les morceaus de code fournis ne sont pas forcément entiers. Nous pensons que les rédacteurs partent du principe que le lecteur a fait une lecture linéaire et dans l'ordre, chose que nous n'avons pas fait.
\end{itemize}

\subsection{Améliorations}
Les améliorations suivantes pourraient être apportées au projet :
\begin{itemize}
    \item L'interface graphique qui est toujours optimisable.
    \item La mise en place d'un cache qui éviterait de réinterroger l'API pour chaque action effectuée et par conséquent limiter le traffic réseau.
    \item Interrogation de l'API Netflix afin de savoir si le film est disponible sur leur plateforme de streaming.
    \item Afficher les films à la une en fonction des préférences de l'utilisateur, par exemple déduites de ses films appréciés ou de ses recherches récentes.
    \item Afficher les informations relatives aux horaires des cinémas les plus proches pour les films actuellement dans les salles.
    \item Notifier les utilisateurs lorsqu'ils sont suivis par un autre utilisateur.
    \item Gestion de plusieurs langues.
\end{itemize}
