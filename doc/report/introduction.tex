
\subsection{Buts}
De nos jours, les médias sont à la portée de tous et accessibles très facilement notamment, grâce aux plateformes telle que Netflix, streaming et bien d'autres encore. Il est donc difficle d'effectuer un choix parmi ces milliers de films mis à disposition. C'est là qu'entre en jeu notre mini-projet que nous avons réalisé dans le cadre du cours de Systèmes d'exploitation mobiles et applications. L'objectif est de développer une application dédiée à la plateforme Android qui propose à ses utilisateurs divers services, tels que la liste des films à la une, obtenir les films similaires à un film, les acteurs concernés et la possibilité de rechercher un film en particulier.
Pour rester dans l'ère du temps, un aspect social est également présent au sein de cette application, l'utilisateur pourra renseigner son appréciation personnelle à propos d'un film, mais également de pouvoir suivre d'autres utilisateurs afin de connaitre leurs appréciations sur différents films qu'ils ont visionnés.
La forme "nous" est utilisée tout au long de ce rapport étant donné que ce projet est réalisé par binôme.

\subsection{Motivations}
Le périmètre de ce projet semblait parfaitement adapté pour exploiter la majorité des techniques et méthodes vues durant le cours de développement d'applications mobiles.
Ce projet est une formidable occasion pour relier la mise en pratique des connaissances acquises en cours et notre passion cinéphile commune. Nous sommes également convaincus que notre entourage (ou un réseau plus large) se servirait volontiers d'une telle application.

\subsection{Méthodologie de travail}
Sur la base d'une analyse préliminaire, nous avons séparé le travail en plusieurs tâches que nous avons assigné à chaque membre du binôme de manière équitable afin d'effectuer le travail en parallèle.
Nous avons adopté une pseudo méthode "agile", en factorisant le projet en petites tâches distinctes et en nous fixant des délais pour les réaliser. Le partage du code s'est fait avec git et gitlab. Nous nous sommes servis des \textit{issues} gitlab pour représenter nos tâches et du "board" du projet pour avoir une vision globale du travail accompli (par qui et quand) et du travail restant.
